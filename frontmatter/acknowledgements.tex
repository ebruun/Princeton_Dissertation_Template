% -------------------------
% ACKNOWLEDGEMENTS
% -------------------------

\acknowledgmenttext{
    This dissertation owes its completion to the invaluable advice, support, and guidance of a wide network of individuals.

    First and foremost, I extend my deepest gratitude to Prof. Sigrid Adriaenssens, my advisor in the Civil and Environmental (CEE) Engineering department. Over the past five years, Prof. Adriaenssens has been instrumental in helping me navigate numerous academic challenges. I am particularly thankful for the trust and academic freedom she afforded me, enabling me to explore a broad range of novel and fascinating topics in the field of construction robotics. Her encouragement to publish early in my PhD, although daunting at the time, has proven invaluable, paving the way for a transition to a faculty position at Georgia Tech later this year. In such a competitive academic landscape, I am grateful for Prof. Adriaenssens' unwavering support and invaluable guidance throughout this process. Thank you also to the members of the Form Finding Lab and all the visiting researchers that worked with our group over the last few years.

    Equally deserving of recognition is my co-advisor, Prof. Stefana Parascho, whose mentorship has been invaluable. Joining her research group in Princeton's School of Architecture (SoA) introduced me to the captivating field of digital fabrication and robotics. Despite my initial unfamiliarity with the field, Prof. Parascho patiently guided me through the fundamentals and provided invaluable research opportunities. Many of the ideas we explored together have culminated in the research forming the core of this dissertation.

    I would also like to thank everybody I interacted with at the SoA's Embodied Computation Lab (ECL) where most of the physical robotic construction work featured in this dissertation took place. Amidst a global pandemic, the ECL was a true refuge. Tucked away in the forest at the edge of campus, it was a place that I came to see as part research laboratory and part oasis. Some of the fondest memories from my PhD are from time spent working in the ECL. It is a beautiful place despite the chaos. In particular, I would like to thank Ian Ting, for the help on all the projects we worked together on, for all the conversations, and for all the smoke breaks. I would also like to thank Bill Tansley for his great attitude and willingness to help with anything.

    I would also like to thank the many graduate student and postdoc friends I made along the way. There are too many to name all of you individually, but know that while the PhD can at times be a lonely endeavor, together you all made it significantly less lonely. Thank you to everyone I spent time with at the Ivy. Our post-pandemic time there were formative — a place to celebrate victories, mourn losses, compete in trivia (team name: Ivy Promises), or simply escape for a while. From my PhD cohort in CEE, I am happy to have made such lasting friendships: Sara, thank you for introducing me to Elena; Aaron, thank you for taking me to some beautiful places to go fishing. Thank you also to everybody on The Dammed, the CEE softball team. Playing in the summer league (and winning the C-league in 2023) was some of the most fun I had at Princeton.

    I am deeply grateful to my parents for their steadfast encouragement and inspiration, serving as my unwavering role models in both life and academia. Their support and advice has been a source of strength throughout my academic journey. To my sister, I extend my gratitude for her example of balancing academic excellence with a fulfilling personal life. To Felix, thank you for being a constant source of joy.

    And most importantly, I would like to thank Elena, my future wife, for supporting me. While I might be leaving Princeton with a PhD, you were by far the best thing that happened to me here. I am so excited about taking the next steps in life together!
}